\documentclass[a4paper,14pt]{article}
\usepackage[T1]{fontenc}
\usepackage[utf8]{inputenc}
\usepackage{lmodern}
\usepackage{amsmath}
\usepackage{amsfonts}
\usepackage{amssymb}
\usepackage{amsthm}
\usepackage{graphicx}
\usepackage{color}
\usepackage{xcolor}
\usepackage{url}
\usepackage{textcomp}
\usepackage{parskip}
\usepackage{pdfpages}
\renewcommand{\figurename}{Slika}
\usepackage{hyperref}
\usepackage{float}
\usepackage{enumitem}
\usepackage{subfig}

\graphicspath{ {./img/} }

\title{Operativni sistem Linux}
\author{Aleksa Siriški}
\date{May 2022}

\begin{document}

\begin{center}
\textbf{Gimnanzija „Jovan Jovanović Zmaj“}
\\
Novi Sad
\end{center}
\vfill
\begin{center}
	\begin{large}
		\textbf{Maturski rad iz Operativnih Sistema}
		\bigskip 
	\end{large}
	\\
	\begin{huge}
        \textbf{Operativni sistem Linux}
	\end{huge}
\end{center}
\vfill
\begin{normalsize}
Profesor mentor:
\hfill
Učenik:
\\
Saša Tošić
\hfill
Aleksa Siriški IV-6
\end{normalsize}
\vfill
\begin{center}
Novi Sad, maj 2022. god.
\end{center}
\newpage

\section{Predgovor}
Za ovu temu sam se opredelio iz više razloga. Prvenstveno zbog ljubavi prema informacionim tehnologijama, koju sam stekao zahvaljujući mojim roditeljima. Drugi razlog je to što smatram da je ovo veoma fascinantna tema, jer obuhvata kompleksnost koje se može postići kada na jednom projektu radi čitav svet. Na kraju, ono što me je privuklo da izaberem baš ovu temu, jeste činjenica da je budućnost IT-a slobodan i besplatan kod.
\\\\
U ovom radu, analiziraću osnovne komponente jednog izuzetnog operativnog sistema, istoriju njegove kreacije kao i filozofski pogled na isti.
\newpage

\renewcommand{\contentsname}{Sadržaj}
\tableofcontents
\newpage

\section{Istorija}
Unix su stvorili i objavili Ken Thompson i Dennis Ritchie 1970. godine. Kasnije je prekucan u C programskom jeziku i time postao veoma fleksibilan i izmenljiv. Razni fakulteti i univerziteti su pravili svoje verzije Unixa, npr. BSD koji je nalik Linuxa i dan danas u upotrebi.
\\\\
Richard Matthew Stallman, osnivač GNU projekta, je uz svoj tim započeo izradu kompletnog operativnog sistema čija je glavna namena da bude otvorena i slobodna alternativa za Unix. Jedina stvar koja je falila jeste kernel, sto je jezgro operativnog sistema koje služi da poveže sve druge komponente u zajednicu.
\\\\
Linus Torvalds koji je bio iznerviran nedostatkom kernela za potpuno slobodan i otvoren operativni sistem odlučuje da napiše svoj sopstveni. Već je bio upućen u Minix i GNU softver te je tokom studija u Finskoj započeo projekat nazvan "Linux". U početku je jedini radio na njemu, ali do danas se priključilo preko 15000 programera u kreiranju kernela koji se sastoji iz vise od 17 miliona linija koda.
\\\\
\begin{figure}[h]
	\centering
    \subfloat[Ken Thompson i Dennis Ritchie, tvorci UNIX-a]{{\includegraphics[height=3cm,width=4cm]{ken-thompson-dennis-ritchie} }}
    \hspace{1cm}
    \subfloat[Linus Torvalds, tvorac Linux-a]{{\includegraphics[height=3cm,width=4cm]{linus-torvalds} }}
\end{figure}
\begin{figure}[h]
	\centering
    \subfloat[Linux]{{\includegraphics[height=3cm,width=4cm]{linux} }}
    \hspace{1cm}
    \subfloat[Richard Matthew Stallman, tvorac GNU-a]{{\includegraphics[height=3cm,width=4cm]{richard-stallman} }}
\end{figure}
\newpage

\section{Komponente}
\newpage

\section{Komande}
\newpage

\section{Hijerarhija sistema datoteka}
\newpage

\section{Radna površina}
\newpage

\section{Zaključak}
\newpage

\section{Literatura}
\newpage

\section{BIOGRAFIJA MATURANTA}
Aleksa Siriški rođen je 10. jula 2003. godine u Novom Sadu. Pohađao je Osnovnu školu „Svetozar Marković Toza“ do šestog razreda. Tamo stiče interesovanje za matematiku, fiziku, informatiku i jezike. Septembra 2016. upisuje Osnovnu školu pri Gimnaziji „Jovan Jovanović Zmaj“, kako bi intenzivnije radio u oblastima matematike, fizike i informatike. Istovremeno pohađa programerski kurs „Centar za mlade talente“. Školovanje nastavlja u istoj gimnaziji i opredeljuje se za smer „Učenici sa posebnim sposobnostima za računarstvo i informatiku“. Naredne četiri godine, uporedo sa školom, pohađa i kurseve engleskog i nemačkog jezika. Planira da više obrazovanje stekne na Prirodno matematičkom fakultetu, smer Informacione tehnologije.
\newpage

\end{document}